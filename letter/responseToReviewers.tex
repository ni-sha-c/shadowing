\documentclass[11pt]{article}
\usepackage[utf8]{inputenc}
\usepackage[margin=1.1in]{geometry}
\usepackage{graphicx}
\usepackage[numbers]{natbib}
\usepackage{verbatim}
\usepackage[textsize=tiny]{todonotes}
\usepackage{amsmath,amssymb}
\usepackage{fancyhdr}
\usepackage[colorlinks=true,linkcolor=blue,urlcolor=blue]{hyperref}
\pagestyle{fancy}
\lhead{Response to the Reviewers}
\rhead{30th October 2020}
\lfoot{\href{mailto:nishac@mit.edu}{Nisha Chandramoorthy} and \href{mailto:qiqi@mit.edu}{Qiqi Wang}}
\cfoot{}
\rfoot{\thepage}
\renewcommand{\headrulewidth}{0.4pt}
\renewcommand{\footrulewidth}{0.4pt}

\definecolor{azure}{rgb}{0.0, 0.5, 1.0}
\definecolor{burgundy}{rgb}{0.5, 0.0, 0.13}
\newcommand{\reviewerOne}[1]{{\color{burgundy}\textbf{#1}}}
\newcommand{\reviewerTwo}[1]{{\color{azure}\textbf{#1}}}




\title{}
\author{}
\date{30th October 2020}
\setlength{\marginparwidth}{2.1cm}
\begin{document}

\begin{center}
		\Large{\textbf{Manuscript JCOMP-D-20-01958: Response to Reviewers}}
\end{center}
\medskip
We would first like to thank both the reviewers for investing their time to read our work carefully and provide us with thoughtful feedback. We have substantially revised our work to incorporate all of both the reviewers' suggestions. In the following letter, we answer each of their comments and hope that both reviewers would find that the 
important issues they raised have been satisfactorily addressed by us.
Quotes from the Reviewer 1's comments are highlighted in \reviewerOne{red}, and those from Reviewer 2's are in \reviewerTwo{blue} in order to distinguish them from our responses to the comments, which immediately follow the comments. Consistently, in our revised manuscript, we highlight the changes inspired by Reviewer 1's and Reviewer 2's comments in \reviewerOne{red} and \reviewerTwo{blue} respectively.
\section{Response to Reviewer 1}
We are very grateful to you for asking thought-provoking
questions on extending our approach to higher-dimensional unstable manifolds 
and on the applications of the CLV self-derivative computations.

1. \reviewerOne{There are some published articles that support some conclusions of this manuscript...All of these should give the authors more information about the topic related to this manuscript}

We thank the reviewer for pointing out multiple references that support the conclusions of the present paper. In our revised manuscript, we have added them to our discussion section (the edits are highlighted in \reviewerOne{red}), where we had originally referred to Clean Numerical Simulation (CNS).


2. \reviewerOne{There is another thing that might be mentioned.  Periodic solutions of some chaotic systems, such as three-body system, have physical meanings.  Although three-body systems are chaotic in essence, periodic orbits of three-body system indeed exist.}

Thank you for pointing this out. In section \ref{sec:nonphysical}, we have included the idea that periodic solutions may be atypical but not necessarily nonphysical.

\section*{Response to Reviewer 2}
We are very grateful to you for recommending edits that have substantially improved our manuscript.

\reviewerTwo{My main technical objection is the claim that the findings here imply that all numerical solutions of chaotic systems might be "unphysical". ...In other words, why must a numerical solution closely follow a specific realization of the exact chaotic system? }

Thank you for your question! Traditionally, the justification for using numerical solutions to model chaotic governing physics is shadowing. While we do not contend that there may be alternative perspectives on numerical solutions, we question this traditional justification, based on the atypicality of shadowing trajectories that we show in the paper. To elaborate on this traditional argument for shadowing justifying the correctness of numerical solution, let us consider a chaotic ODE of the form 
\begin{align}
    \dfrac{d x}{dt} = f(x, s),
\end{align}
where $x \in \mathbb{R}^d$ is a $d$-dimensional state vector, $s$ is a set of
parameters in $\mathbb{R}^p$, and $f:\mathbb{R}^d \times \mathbb{R}^p \to \mathbb{R}^d$ is the generating flow. A numerical solution of the system may be represented as $\hat{x}(t)$ such that 
\begin{align}
    \dfrac{d\hat{x}}{dt} = f(\hat{x}, s) + \delta f(\hat{x},s), 
\end{align}
where $\delta f:\mathbb{R}^d \times \mathbb{R}^p \to \mathbb{R}^d$ is the model discrepancy introduced due to numerical discretization. A corollary of the shadowing lemma is that for every $\epsilon > 0,$ there exists  $\sup_{x} \|\delta f(x,s)\|$ is small, 

\end{document}
