\documentclass{letter}
\title{Letter to the Editor}
\begin{document}
\hspace{1.6in} 
\begin{large}{Significance and novelty}\end{large}
\\
\\

The major contribution of this work is to show, through rigorous examples, that shadowing solutions can be nonphysical, and to expose the implications of this finding. That is, we show that long-time averages computed along shadowing orbits are often drastically different from ensemble averages. This finding dispels the notion that small parameter perturbations in
a chaotic system result in small differences in its statistics or long-time averages. Hence, the trustworthiness of numerical simulations of chaos, which are small perturbations of the true governing equation, in representing physically observed long-term behavior, is called into question. This finding is especially important in climate studies and turbulent flows in engineering, where detailed chaotic simulations are used for statistical predictions. It also hints at the theoretical possibility of harnessing this distinctive feature of chaotic dynamics: it implies that we can favorably alter climate statistics significantly by applying subtle perturbations. Finally, the nonphysicality of shadowing also reveals that shadowing-based sensitivity computation of parametric derivatives of statistics can lead to incorrect results. 


Nisha Chandramoorthy and Qiqi Wang.
\end{document}
