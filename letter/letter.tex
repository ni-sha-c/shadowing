\documentclass{letter}
\title{Letter to the Editor}
\begin{document}
\hspace{1.6in} 
\begin{large}{Letter to the Editor}\end{large}
\\
\\
Dear Editor,

We are happy to submit our manuscript titled 
``On the probability of finding a nonphysical solution through shadowing ''
for consideration for publication in the Journal of Computational Physics. Thank you very much for giving us the 
opportunity to have our work reviewed!

The purpose of this work is to raise an important question on the reliability of numerical simulations of chaos that resonates across all scientific disciplines. We ask whether perturbed solutions of a 
governing equation reproduce its physical behavior. This is a question distinct from whether perturbed solutions of a chaotic system lie close to a true solution. This latter question has historically been answered in the affirmative, at least in hyperbolic dynamical systems, by shadowing. In this paper, we show that while
perturbed solutions such as a numerical simulation, 
may have their corresponding true solutions that shadow them, these shadowing solutions can be \emph{nonphysical}. We define the concept of nonphysical solution as a true solution to a governing equation along which long-time averages do not converge to expectations with respect to the stationary probability distribution on the attractor. Equivalently, the infinite-time averages along a nonphysical solution do not converge to the same value as that along almost every solution.

We argue, through analytical constructions, that shadowing solutions are typically nonphysical solutions. We show that the probability of finding a nonphysical shadowing solution can be 1. This has the implication that perturbed solutions such as numerical solutions may be close to nonphysical solutions, and hence produce incorrect long-term or statistical behavior. In addition, sensitivity analysis methods based on the concept of shadowing also give incorrect results for the response of statistics to small perturbations. We construct an example to show that 
there are chaotic systems in which small parameter perturbations can, in fact, cause a dramatic change in the statistics. This extreme non-smooth response of a chaotic attractor further strengthens our conclusion that the assumption of physicality of shadowing solutions is easily violated.

This paper raises a fundamental and profound issue with numerical simulation and sensitivity analysis of chaotic governing equations. It therefore has immediate implications for the entire computational science and engineering community, which must work together to rethink chaotic simulations. 
Thank you very much for your time and consideration,
\\
\\
Nisha Chandramoorthy and Qiqi Wang.
\end{document}
