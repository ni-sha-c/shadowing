\section{Supplementary material}
\subsection{The Lorenz Map}
\subsection{Approximation of the Lorenz Map}

\subsubsection{Computing shadowing solution of the Lorenz map}
In this section, we discuss how to numerically compute the shadowing trajectories of the Lorenz map. One of the motivations for approximating the Lorenz map is that having a closed form expression is necessary in the algorithm we discuss for numerical shadowing solution computation. The program used to generate Figure 
\ref{fig:lorenz_params_scaled} is in \verb+lorenz_map/shadow.py+. The main logic behind the numerical procedure to compute the shadowing solution can be found in that script, within the function \verb+shadow+, which calls \verb+find2+. Let $x_n, n = 0,1,2,3\cdots,N$ be a given perturbed solution given to us until time $N$. The goal is to find a sequence $\left\{ y_n\right\}_{n=0}^{N}$ that is a shadowing solution. This means that $y_n$ must satisfy the governing equation: $y_{n+1} = \varphi(y_n),$ where $\varphi$ is the Lorenz map, and, 
$y_n$ and $x_n$ must stay close to each to other for the length of time, $N$. We find the sequence $y_n$ backward in time. Let $z_{\rm sep,s}$ be the $z$-location of the cusp of the perturbed Lorenz map that generated $x_n$; to indicate the original map, we omit the $s$ in the subscript. We split the domain into two regions, $[z_{\rm min}, z_{\rm sep}]$ and $(z_{\rm sep}, z_{\rm max}],$ and 
obtain, for $n = N-1, \cdots, 0,$
\begin{align}
    y_n = \begin{cases}
                     z_{\rm sep} + f(y_{n+1}), & x_n > z_{\rm sep, s} \\
                     z_{\rm sep} - f(y_{n+1}), & x_n \leq z_{\rm sep, s}.
    \end{cases}
    \label{eqn:shadowingFormula}
\end{align}
Here the function $f$ is implemented in 
\verb+find2+. Inside \verb+find2+, given a $y$, 
a recursive procedure is used to 
find a number $v_k = f(y_{n+1})$ such that if $y_n$ set 
using Eq. \ref{eqn:shadowingFormula} is (approximately) the shadowing 
solution. At $n = N-1,$ $y_N$ is initialized with a random guess close to $x_N.$
Then, the following recursive equation is solved to compute $f(y_{n+1})$.
\begin{align}
        v_{k+1} = v_k \;  \exp{\big({y_{n+1} - \varphi(y_{n,k})}\big)} \; \exp{\Big(\dfrac{-1}{\varphi'(y_{n,k}) v_k}\Big)}, \; k = 0,1,\cdots,
\end{align}
At $k=0,$ we randomly assign $v_0 = 1,$ and continue the above iteration 
until $|v_k|$, and $|y_{n+1} - \varphi(y_{n,k})|$ are both less than 
a tolerance. When the iteration converges, at some $k$,
$f_i(y_{n+1}) = v_k.$ Notice that by construction, the
sequence $\left\{ y_n\right\}$ computed this way is a shadowing solution.