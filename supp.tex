\section{Supplementary material}
%\subsection{The Lorenz Map}
\subsection{Approximation of the Lorenz Map}
\label{sec:lorenzMapApprox}
The motivation for approximating the Lorenz map is that a 
closed form expression for the map is necessary for our numerical shadowing procedure. In a small region around the cusp, we approximate the map using an exponential function. The tails on both sides are fitted with a sum of a cubic polynomial and a rational function. Thus, the approximate Lorenz map has the 
following closed form expression:
\begin{align}
    \varphi(x) = \begin{cases}
        z_{\rm max} - f_R(x - z_{\rm sep})   &  x > z_{\rm sep} \\
        z_{\rm max} -  f_L(z_{\rm sep} - x)   &  x \leq z_{\rm sep}
    \end{cases}
\end{align}
where 
\begin{align}
    f_R(y) = \Big( (1000 y)^p + 
        \sum_{n=0}^3 p_{R,n} y^n + \dfrac{a_{L,0} + a_{L,1} y}{\sum_{n=0}^3 b_{L,n} y^n}\Big),
\end{align}
and,
\begin{align}
    f_L(y) = \Big((1000 y)^p 
        + \sum_{n=0}^3 p_{L,n} y^n + \dfrac{a_{R,0} + a_{R,1}y}{\sum_{n=0}^3 b_{R,n} y^n}\Big).
\end{align}
The location of the cusp is denoted $z_{\rm sep}$ (in which ``sep'' stands 
for \emph{separation}), and the maximum and minimum values encountered 
in the Lorenz map iterates are denoted $z_{\rm max}$ and $z_{\rm min}$, respectively. The exponent of the cusp is denoted $p$. The coefficients of the cubic polynomial modelling the left (right) tail are  denoted $p_{L,n}, n = 0,1,2,3$ ($p_{R,n}, n = 0,1,2,3$). The coefficients of the numerator and denominator of the rational function modelling the left (right) tail are denoted $a_{L,0}, a_{L,1}$ ($a_{R,0}, a_{R,1}$) and $b_{L,n}, n=0,1,2,3$ ($b_{R,n}, n=0,1,2,3$), respectively. The values of these coefficients are obtained by 
regression on the map at standard parameters are shown in the table below.
\begin{table}[H]
    \centering
    \begin{tabular}{|c|c|}
    \hline 
         $z_{\rm sep}$ = 38.55302437476555 &
         $z_{\rm min}$ =  29.213182255013322 \\ 
         $z_{\rm max}$ =  47.978140718671284 & 
         $p$ = 0.28796740575434676   \\
         $p_{L,3}$ = -0.00024683786275242047 & $p_{L,2} = $ 
         0.016174566354858824 \\
         $p_{L,1} = $ 0.40179772568004946 & 
         $p_{L, 0}  = $ -0.24612651488351725 \\
         $p_{R,3} = $ -0.00020712463321688308 & 
         $p_{R,2} = $ 0.017130843276711716 \\
         $p_{R, 1} = $ 0.3930080703420676 & 
         $p_{R, 0} = $ -0.23471384266765036 \\
         $a_{L,1}$ = -0.05405742075580959 & $a_{L,0}$ = -0.05405742075580959 \\
         $a_{R,1}$ = -0.05351127783496397 & $a_{R,0}$ = 0.22489891059122896 \\
         $b_{L,3}$ = 0.5609397451213353 & $b_{L,2}$ = -0.3491184293228338 \\  $b_{L,1}$ = 2.419972619058592 & $b_{L,0}$ = 1.0 \\
         $b_{R,3}$ = 0.6456076059873844 & $b_{R,2}$ = -0.34840383986411055 \\ $b_{R,1}$ = 2.6035438510917692 &  $b_{R,0}$ = 1.0 \\
         \hline
    \end{tabular}
    \caption{The fitting parameters of the Lorenz map at $\sigma = 10, \beta = 8/3$ and $\rho = 28.$}
    \label{tab:my_label}
\end{table}

\begin{comment}
\subsubsection{Computing shadowing solutions of the Lorenz map}
In this section, we discuss how to numerically compute the shadowing trajectories of the Lorenz map.  The program used to generate Figure 
\ref{fig:lorenz_params_scaled} is in \verb+lorenz_map/shadow.py+. The main logic behind the numerical procedure to compute the shadowing solution can be found in that script, within the function \verb+shadow+, which calls \verb+find2+. Let $x_n, n = 0,1,2,3,\cdots,N$ be a perturbed solution given to us until time $N$. The goal is to find a sequence $\left\{ y_n\right\}_{n=0}^{N}$ that is a shadowing solution. This means that $y_n$ must satisfy the governing equation: $y_{n+1} = \varphi(y_n),$ where $\varphi$ is the Lorenz map, and, 
$y_n$ and $x_n$ must stay close to each to other for the length of time, $N$. We find the sequence $y_n$ backward in time. Let $z_{\rm sep,s}$ be the $z$-location of the cusp of the perturbed Lorenz map that generated $x_n$; to indicate the original map, we omit the $s$ in the subscript. We split the domain into two regions, $[z_{\rm min}, z_{\rm sep}]$ and $(z_{\rm sep}, z_{\rm max}],$ and 
obtain, for $n = N-1, \cdots, 0,$
\begin{align}
    y_n = \begin{cases}
                     z_{\rm sep} + f(y_{n+1}), & x_n > z_{\rm sep, s} \\
                     z_{\rm sep} - f(y_{n+1}), & x_n \leq z_{\rm sep, s}.
    \end{cases}
    \label{eqn:shadowingFormula}
\end{align}
Here the function $f$ is implemented in 
\verb+find2+. Inside \verb+find2+, given a $y_{n+1}$, 
a recursive procedure is used to 
find a number $v_k = f(y_{n+1})$ such that $y_n$ set 
using Eq. \ref{eqn:shadowingFormula} is (approximately) the shadowing 
solution. At $n = N-1,$ $y_N$ is initialized with a random guess close to $x_N.$
Then, the following recursive equation is solved to compute $f(y_{n+1})$.
\begin{align}
        v_{k+1} = v_k \;  \exp{\big({y_{n+1} - \varphi(y_{n,k})}\big)} \; \exp{\Big(\dfrac{-1}{\varphi'(y_{n,k}) v_k}\Big)}, \; k = 0,1,\cdots,
\end{align}
At $k=0,$ we randomly assign $v_0 = 1,$ and continue the above iteration 
until $|v_k|$, and $|y_{n+1} - \varphi(y_{n,k})|$ are both less than 
a tolerance. When the iteration converges, at some $k$,
$f(y_{n+1}) = v_k.$ Notice that by construction, the
sequence $\left\{ y_n\right\}$ computed this way solves the governing equation.
To conclude that the sequence approximates the shadowing solution, we only need to check that $x_n$ and $y_n$ are close to each other for all times up to $N.$
\end{comment}